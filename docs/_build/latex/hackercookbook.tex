%% Generated by Sphinx.
\def\sphinxdocclass{report}
\documentclass[letterpaper,10pt,english]{sphinxmanual}
\ifdefined\pdfpxdimen
   \let\sphinxpxdimen\pdfpxdimen\else\newdimen\sphinxpxdimen
\fi \sphinxpxdimen=.75bp\relax

\PassOptionsToPackage{warn}{textcomp}
\usepackage[utf8]{inputenc}
\ifdefined\DeclareUnicodeCharacter
% support both utf8 and utf8x syntaxes
  \ifdefined\DeclareUnicodeCharacterAsOptional
    \def\sphinxDUC#1{\DeclareUnicodeCharacter{"#1}}
  \else
    \let\sphinxDUC\DeclareUnicodeCharacter
  \fi
  \sphinxDUC{00A0}{\nobreakspace}
  \sphinxDUC{2500}{\sphinxunichar{2500}}
  \sphinxDUC{2502}{\sphinxunichar{2502}}
  \sphinxDUC{2514}{\sphinxunichar{2514}}
  \sphinxDUC{251C}{\sphinxunichar{251C}}
  \sphinxDUC{2572}{\textbackslash}
\fi
\usepackage{cmap}
\usepackage[T1]{fontenc}
\usepackage{amsmath,amssymb,amstext}
\usepackage{babel}



\usepackage{times}
\expandafter\ifx\csname T@LGR\endcsname\relax
\else
% LGR was declared as font encoding
  \substitutefont{LGR}{\rmdefault}{cmr}
  \substitutefont{LGR}{\sfdefault}{cmss}
  \substitutefont{LGR}{\ttdefault}{cmtt}
\fi
\expandafter\ifx\csname T@X2\endcsname\relax
  \expandafter\ifx\csname T@T2A\endcsname\relax
  \else
  % T2A was declared as font encoding
    \substitutefont{T2A}{\rmdefault}{cmr}
    \substitutefont{T2A}{\sfdefault}{cmss}
    \substitutefont{T2A}{\ttdefault}{cmtt}
  \fi
\else
% X2 was declared as font encoding
  \substitutefont{X2}{\rmdefault}{cmr}
  \substitutefont{X2}{\sfdefault}{cmss}
  \substitutefont{X2}{\ttdefault}{cmtt}
\fi


\usepackage[Bjarne]{fncychap}
\usepackage{sphinx}

\fvset{fontsize=\small}
\usepackage{geometry}


% Include hyperref last.
\usepackage{hyperref}
% Fix anchor placement for figures with captions.
\usepackage{hypcap}% it must be loaded after hyperref.
% Set up styles of URL: it should be placed after hyperref.
\urlstyle{same}
\addto\captionsenglish{\renewcommand{\contentsname}{Contents:}}

\usepackage{sphinxmessages}
\setcounter{tocdepth}{1}



\title{Hacker Cook Book}
\date{May 09, 2020}
\release{0.1.0}
\author{@theDevilsVoice}
\newcommand{\sphinxlogo}{\vbox{}}
\renewcommand{\releasename}{Release}
\makeindex
\begin{document}

\pagestyle{empty}
\sphinxmaketitle
\pagestyle{plain}
\sphinxtableofcontents
\pagestyle{normal}
\phantomsection\label{\detokenize{index::doc}}


\noindent{\hspace*{\fill}\sphinxincludegraphics{{hacked}.jpeg}\hspace*{\fill}}


\chapter{The Hacker Cookbook}
\label{\detokenize{_source/introduction:the-hacker-cookbook}}\label{\detokenize{_source/introduction::doc}}
This is a collaborative effort. The goal is to collect recipes for
an eclectic mix of dishes from around the hacker community. Tell
us about your favorite recipe or something you like to make with
your own unique twist. Feel free to include a recommendation for
booze pairing.

General questions should be directed to \sphinxhref{https://twitter.com/mzbat}{@mzbat} or \sphinxhref{https://twitter.com/iheartmalwar}{@iheartmalware} .

We can’t promise that all submissions will be included in the final
printed book. We do plan to leave this repo up indefinitely. We
intend to give any proceeds to a charitable cause.


\chapter{Submissions}
\label{\detokenize{_source/submissions:submissions}}\label{\detokenize{_source/submissions::doc}}\begin{enumerate}
\sphinxsetlistlabels{\arabic}{enumi}{enumii}{}{.}%
\item {} 
Fork this repository. Create a subdirectory for your recipe. Try to make the name of the subdirectory match the name of your recipe.

\item {} 
Include a .md document in your sub dir {[}Github Markdown Cheatsheet{]} (\sphinxurl{https://github.com/adam-p/markdown-here/wiki/Markdown-Cheatsheet}) You can use the {[}template file{]}(\sphinxurl{https://github.com/Nocsetse/1337-Noms-The-Hacker-Cookbook/blob/master/template.md}) as a guide to help you in your effort. There is a {[}markdown tutorial here{]}(\sphinxurl{http://www.markdowntutorial.com/})

\item {} 
Please include one or two pictures of the preparation or completed dish you think would be helpful. Keep it tasteful, save all that goatse stuff for your next lemon party.

\item {} 
Don’t forget to add your name/handle to the bottom of the {[}credits file{]}(\sphinxurl{https://github.com/Nocsetse/1337-Noms-The-Hacker-Cookbook/blob/master/credits.md})

\item {} 
Do a pull request against this repo and we will merge your stuff in. Reach out to {[}@theDevilsVoice{]}(\sphinxurl{https://twitter.com/thedevilsvoice}) if you need technical help.

\end{enumerate}

NOTE: You are releasing the rights to your work under the {[}included LICENSE{]}(\sphinxurl{https://github.com/Nocsetse/1337-Noms-The-Hacker-Cookbook/blob/master/license.md})


\section{Who You Are}
\label{\detokenize{_source/submissions:who-you-are}}\begin{itemize}
\item {} 
Include how you’d like to be credited/identified in your .md file.

\item {} 
Be sure you add your details to the end of the {[}credits.md{]}(\sphinxurl{https://github.com/Nocsetse/1337-Noms-The-Hacker-Cookbook/blob/master/credits.md}) file.

\end{itemize}


\section{Name of dish}
\label{\detokenize{_source/submissions:name-of-dish}}\begin{itemize}
\item {} \begin{description}
\item[{Just like it sounds, we have to call the dish something.}] \leavevmode\begin{itemize}
\item {} 
Choose a category (also OK to add a new category)

\item {} 
Create a folder to hold your files.

\item {} 
Name it something like “myname\_dishname”

\end{itemize}

\end{description}

\end{itemize}


\section{Ingredients}
\label{\detokenize{_source/submissions:ingredients}}\begin{itemize}
\item {} \begin{description}
\item[{Please include Metric and English wherever possible}] \leavevmode\begin{itemize}
\item {} 
{[}Conversion to/from English/Metric Measurements{]}(\sphinxurl{http://www.sciencemadesimple.com/volume\_conversion.php})

\end{itemize}

\end{description}

\item {} \begin{description}
\item[{Make a file with the same format as the parent folder, but with the .md extension}] \leavevmode\begin{itemize}
\item {} 
“myname\_dishname.md”

\end{itemize}

\end{description}

\item {} 
Set of steps that explain how to create your dish go in your .md file.

\end{itemize}


\section{Photos}
\label{\detokenize{_source/submissions:photos}}\begin{itemize}
\item {} 
Instructional and/or completed dish should be uploaded to your sub\sphinxhyphen{}folder in the repository.

\end{itemize}


\section{How should the Dish Be served?}
\label{\detokenize{_source/submissions:how-should-the-dish-be-served}}\begin{itemize}
\item {} 
Temperature

\item {} 
Serving tips

\item {} 
Beer, wine or drink pairings.

\item {} 
Anything that gives it the finishing touch.

\end{itemize}



\renewcommand{\indexname}{Index}
\printindex
\end{document}